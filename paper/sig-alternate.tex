% This is "sig-alternate.tex" V2.0 May 2012
% This file should be compiled with V2.5 of "sig-alternate.cls" May 2012
%
% This example file demonstrates the use of the 'sig-alternate.cls'
% V2.5 LaTeX2e document class file. It is for those submitting
% articles to ACM Conference Proceedings WHO DO NOT WISH TO
% STRICTLY ADHERE TO THE SIGS (PUBS-BOARD-ENDORSED) STYLE.
% The 'sig-alternate.cls' file will produce a similar-looking,
% albeit, 'tighter' paper resulting in, invariably, fewer pages.
%
% ----------------------------------------------------------------------------------------------------------------
% This .tex file (and associated .cls V2.5) produces:
%       1) The Permission Statement
%       2) The Conference (location) Info information
%       3) The Copyright Line with ACM data
%       4) NO page numbers
%
% as against the acm_proc_article-sp.cls file which
% DOES NOT produce 1) thru' 3) above.
%
% Using 'sig-alternate.cls' you have control, however, from within
% the source .tex file, over both the CopyrightYear
% (defaulted to 200X) and the ACM Copyright Data
% (defaulted to X-XXXXX-XX-X/XX/XX).
% e.g.
% \CopyrightYear{2007} will cause 2007 to appear in the copyright line.
% \crdata{0-12345-67-8/90/12} will cause 0-12345-67-8/90/12 to appear in the copyright line.
%
% ---------------------------------------------------------------------------------------------------------------
% This .tex source is an example which *does* use
% the .bib file (from which the .bbl file % is produced).
% REMEMBER HOWEVER: After having produced the .bbl file,
% and prior to final submission, you *NEED* to 'insert'
% your .bbl file into your source .tex file so as to provide
% ONE 'self-contained' source file.
%
% ================= IF YOU HAVE QUESTIONS =======================
% Questions regarding the SIGS styles, SIGS policies and
% procedures, Conferences etc. should be sent to
% Adrienne Griscti (griscti@acm.org)
%
% Technical questions _only_ to
% Gerald Murray (murray@hq.acm.org)
% ===============================================================
%
% For tracking purposes - this is V2.0 - May 2012

\documentclass{sig-alternate}

\begin{document}
%
% --- Author Metadata here ---
%\conferenceinfo{WOODSTOCK}{'97 El Paso, Texas USA}
%\CopyrightYear{2007} % Allows default copyright year (20XX) to be over-ridden - IF NEED BE.
%\crdata{0-12345-67-8/90/01}  % Allows default copyright data (0-89791-88-6/97/05) to be over-ridden - IF NEED BE.
% --- End of Author Metadata ---

\title{Twitter Sentiment Analysis}
\subtitle{Advanced Internet Computing}

%
% You need the command \numberofauthors to handle the 'placement
% and alignment' of the authors beneath the title.
%
% For aesthetic reasons, we recommend 'three authors at a time'
% i.e. three 'name/affiliation blocks' be placed beneath the title.
%
% NOTE: You are NOT restricted in how many 'rows' of
% "name/affiliations" may appear. We just ask that you restrict
% the number of 'columns' to three.
%
% Because of the available 'opening page real-estate'
% we ask you to refrain from putting more than six authors
% (two rows with three columns) beneath the article title.
% More than six makes the first-page appear very cluttered indeed.
%
% Use the \alignauthor commands to handle the names
% and affiliations for an 'aesthetic maximum' of six authors.
% Add names, affiliations, addresses for
% the seventh etc. author(s) as the argument for the
% \additionalauthors command.
% These 'additional authors' will be output/set for you
% without further effort on your part as the last section in
% the body of your article BEFORE References or any Appendices.

\numberofauthors{4} %  in this sample file, there are a *total*
% of EIGHT authors. SIX appear on the 'first-page' (for formatting
% reasons) and the remaining two appear in the \additionalauthors section.
%
\author{
% You can go ahead and credit any number of authors here,
% e.g. one 'row of three' or two rows (consisting of one row of three
% and a second row of one, two or three).
%
% The command \alignauthor (no curly braces needed) should
% precede each author name, affiliation/snail-mail address and
% e-mail address. Additionally, tag each line of
% affiliation/address with \affaddr, and tag the
% e-mail address with \email.
%
% 1st. author
\alignauthor
Martin Kaufleitner\\
       \affaddr{Vienna University of Technology}\\
       \affaddr{Austria}\\
       \email{e1027229@student.tuwien.ac.at}
% 2nd. author
\alignauthor
Martin Kaufleitner\\
       \affaddr{Vienna University of Technology}\\
       \affaddr{Austria}\\
       \email{e1027229@student.tuwien.ac.at}
\and
% 3rd. author
\alignauthor
Martin Kaufleitner\\
       \affaddr{Vienna University of Technology}\\
       \affaddr{Austria}\\
       \email{e1027229@student.tuwien.ac.at}
\and  % use '\and' if you need 'another row' of author names
% 4th. author
\alignauthor
Martin Kaufleitner\\
       \affaddr{Vienna University of Technology}\\
       \affaddr{Austria}\\
       \email{e1027229@student.tuwien.ac.at}
}
% There's nothing stopping you putting the seventh, eighth, etc.
% author on the opening page (as the 'third row') but we ask,
% for aesthetic reasons that you place these 'additional authors'
% in the \additional authors block, viz.
%\additionalauthors{Additional authors: John Smith (The Th{\o}rv{\"a}ld Group,
%email: {\texttt{jsmith@affiliation.org}}) and Julius P.~Kumquat
%(The Kumquat Consortium, email: {\texttt{jpkumquat@consortium.net}}).}
%\date{30 July 1999}
% Just remember to make sure that the TOTAL number of authors
% is the number that will appear on the first page PLUS the
% number that will appear in the \additionalauthors section.

\maketitle
\begin{abstract}
Twitter Sentiment Analysis is a crucial task in today's scenarios where opinions gain more weight for further investigations and developments of companies. In this paper we want to provide an overview of the state-of-the-art regarding sentiment analysis of Twitter messages, including the extensive growth and the new beneficial possibilities to classify messages of the Twitter platform. The experimental evaluation of our dataset,  its classification results and findings do not contradict to any existing results from other scientific work.
\end{abstract}

\section{Introduction}
Social media has revolutionised the way in which people communicate. Gaining information from social networks is beneficial for analysis of user opinion for example measuring the feedback on a recently released product, looking at opinions concerning upcoming elections or the enjoyment of an ongoing event. 
Sentiment analysis is a relatively new area, which deals with extracting user opinion. Since Twitter has well established as social media platform for personal continuous news and opinions flow from all over the world, it also became an eye catcher for different organisations to get feedback of their product applying sentiment analysis on tweets. As a consequence this trend influenced various marketing and sales stragies and created a new market for companies that are specialise on sentimental analysing tweets, like … An example of a positive sentiment is, „traveling is fun“ alternatively, a negative sentiment is „it’s a horrible day, i am not going outside“. Furthermore, emoticons such as „:-)“ is a postive and „:-(„ is a negative expression which also influence the result of sentiment analyse. Objective texts are deemed not to be expressing any sentiment, such as news headlines, for example “company shelves wind sector plans”.
There are many ways in which data can be leveraged to give a better understanding of user opinion. Sentiment analysis of tweets aims for analysing whether the global opinion located at Twitter concerning an expression is positive or negative. This can be done in different ways, e.g. different grammatic and language rules. Such problems are at the heart of natural language processing (NLP) and data mining research.
The rest of the paper is organized as follows. In Section 2, etc.

\section{Data Description}

The analayzed text posts - so called tweets - are composed by twitter users. They are characterized by a maximum length of 140 characters, including links, pictures and special tagged words. To tag a word certain prefix is needed that is also supported by twitter. For example the @-symbol is used to mention some text represented as a link to another user of twitter. The hash-symbol labels a word to mark important keywords in tweets. 
It�s obvious that collecting many tweets in some particular fields by the Twitter API also take some problems into account: language�s specifications like grammar or slangs. Therefore we focused in our research on the English language.


\section{Pre-Processing}
Since all different kinds of classifiers try to extract, evaluate and interpret different features, the amount and quality of those features is essential for a good result of the classification. We want to examine some state of the art pre-processing and feature reduction techniques, which presented themselves as very promising and effective. While examining them, we will discuss, which ones of the techniques we have also implemented in our Sentiment Analyzing Software and how we did this.

\subsection{Twitter Domain}
Since we are dealing with twitter messages, we have to deal with some specialties, which need some extra preparation during the pre-processing and feature reduction steps. This specialities are stated in almost all papers, which deal with twitter sentiment check, so we want to have a look at them. We only consider the most significant specialities, a more detailed description can be found e.g. et al [1]

\begin{itemize}
\begin{item}
Limited Length\\
Twitter messages have a limited length of 140 characters. This fact leads to the occurrence of many abbreviations and slang formulations. Some examples are "OMG" for "Oh my god" or "FTW" for "For the win".
\end{item}


\begin{item}
Emoticons\\
Also the occurrence of emoticons needs special treatment. There are different approaches, how to take them into account which we want to discuss.
\end{item}

\begin{item}
Casual language\\
Since there are no guidelines or rules how to formulate tweets, they contain very casual language. An example would be a tweet like “OMG I’m soooo huuuuuungry!!!!”. The multiple occurrence of letters in the words “soooo” and “huuuuuungry” would blow up the amount of features, since “hungry”, “huungry”, “huuungry” and so on would all be treated as different words. Also misspelled words lead to the same problem. We will discuss some ways how to address this problem.
\end{item}

\begin{item}
Links\\
Many tweets contain links to other web pages or articles. Those links can mislead the interpretation of a tweet.
\end{item}

\begin{item}
Usernames, Re-Tweets and Hashtags\\
Symbols like RT for Retweet, @ for linking other users or a hashtag often occur in tweets and therefore ask for special treatment.
\end{item}
\end{itemize}

\subsection{Methods}
Now let us have a look at some different methods according to the pre-processing step which deal with the problems occurring in the stated twitter domain.


\subsubsection{Usernames, Hashtags, Links}
Most approaches like [5] remove all usernames, hashtags and links and replace their occurrences with according tags like <username>. This very simple method reduces the amount of features enormously, since otherwise, each username, which does not give any information about a sentiment, would be treated as a feature. Furthermore, all tweets, which are re-tweets are deleted, since otherwise the original tweet would be over weighted and maybe falsify the result. These methods are stated in almost all twitter sentiment analyzing approaches and since they are very simple, we implemented all of them. But despite of the described processing in [2] we did not tag tweets when we e.g. remove a link. Unfortunately the paper does not describe in detail, how they use this tag information afterwards, so we decided to leave it out.


\subsubsection{Emoticons}
Regarding emoticons, there are different approaches which could be followed. The first one doesn’t treat them in any special way, so they are considered like any other word and are listed in the features e.g. [3]. The first basic step is to classify them into e.g. positive and negative ones and then replace them. For example the emoticons :)  :D and XD are all replaced with <positiveEmoticon> and :( :’( and so on are replaced with <negativeEmoticon>. This helps reducing features, but if you let emoticons in your data, they may get over fitted, that means, that the occurrence of an emoticon will almost exclusively decide on the classification of the tweet. This is especially the case, when you retrieved your training data by searching for tweets with emoticons as noisy labels. Therefore there are approaches, like [5], which delete the emoticons completely from the training data and only allow them during classification or some approaches which even remove them from the test data. Since we obtained our test set classification by querying twitter for tweets containing :) for positive and :( for negative, we had to delete them from our training set to prevent over fitting. For the test data, we replaced them with the according positive and negative tags in order to reduce features and then treated them as a normal feature, since removing them would result in a loss of important sentiment information.

\subsubsection{Casual language and misspelled words}
The problem with casual language and misspelled words is, that the number of features grows very vast. To get rid of misspelled words, spell checking algorithms are available, which try to find the intended meaning of a word and replace them. According to words like "huuuungry" there are again two different approaches. Either all multiply occurrences are simplified to at most two letters ("huungry") and then treated as an own feature [1], since multiple letters may express the sentiment stronger, or they are then spell-checked and replaced by the corresponding word like "hungry" in this case. Since we think, that the additional information, given by using multiple letters should not be lost, we implemented the first approach and therefore the @TODO: Keine ahnung wie wir das gemach haben – Also putting all letters in lower case helps reducing unnecessary features.

\subsubsection{Stopwords}
So calledstop words like "and", "the", "for", "a" and so on do not provide any information about the sentiment of a tweet and are therefore unnecessary features which can be removed as proposed in almost all papers. Databases containing this words can be found in the internet.

\subsubsection{Abbreviations}
Abbreviations could either be replaced by their original words, or used as their own feature. Most paper describe the replacement and therefore we also followed this strategy in our project. We used a simple Abbreviations database, tuned it, and then replaced all abbreviations according to this list.

\subsubsection{Tokenization}
After preparing the data, the text has to be separated into different features. We decided to split the sentences at white spaces and punctuations using a Twitter Tokenization library. We did not consider special words like "Don’t" or "I’ll" to stay together, as proposed et al [4].


\section{Classification}

\subsection{Overview}

In the context of sentiment analysis, classification is intended to determine the sentiment of a piece of text (such as a Twitter status), e.g. how positive or negative the sentiment behind the content is. This is also referred to as "sentiment polarity classifications", to differentiate it from other sorts of sentiment. A variety of other metrics exists, such as political affiliation (e.g. conservative/liberal) or various kinds of approval/disapproval \cite{pang2008opinion}. Sentiment classification is a classical natural language processing (NLP) problem, and its high complexity is largely founded in the complexity and ambiguity of human speech itself (cite???).



The already high complexity of sentiment analysis is compounded by the typical terseness of tweets (thus containing few relevant words), as well as frequent use of informal language and (sometimes ambiguous) abbreviations \cite{kouloumpis2011twitter}. Sentiment analysis at the phrase level is more complicated than at the document level, and has been developed only more recently. Furthermore, spelling errors and informal spelling are frequent. \cite{agarwal2011sentiment}.

Furthermore, tweets often combine different or even diametrically opposing sentiments in close proximity. While these are easily separated for humans, this task is very complex for machines \cite{vinodhini2012sentiment}.

On the other hand, microblog content might also offer some clues which are unlikely to appear in a more formal medium. For example, emoticons are frequently used to aid classification \cite{agarwal2011sentiment, kouloumpis2011twitter}, or to tag an unclassified training set \cite{go2009twitter}.




In the literature, many approaches to classifying tweets using various combinations of features and machine learning algorithms are described.

Generally speaking, sentiment analysis poses two separate challenges, which are both critical for the quality of the results. The first is to recognize and extract features which are suitable indicators of sentiment. The second part consists of passing them -- as part of either training or evaluation -- to some mechanism for machine learning in order to assign a single sentiment value to a piece of text.


\subsubsection{Features}

A variety of features can be extracted from messages and used for classification purposes. Unigrams, bigrams and larger $n$-grams represent words or other sorts of tokens (e.g. emoticons and exclamation marks). Sentiments can be assigned to individual words using special dictionaries and used as features \cite{kouloumpis2011twitter}.


In addition to simple lists, processing features as part of more complex structures is also possible. In \cite{agarwal2011sentiment}, a tree kernel is used to perform classification over many different features at once. These features include integers, reals and booleans, and form a hierarchical structure (the tree). A Support Vector Machine is used for learning, with the results outperforming a second implementation using unigrams. Even better results were obtained by combining both.


A fundamental problem of all sentiment analysis methods is that the meaning of words often depends strongly on context \cite{vinodhini2012sentiment}, therefore simple features may be strongly misleading. The simplest example is negation using a preceding "not", which completely inverts the sentiment of the word it is connected to.

Approaches to solving this problem include the use of bigrams, unigrams with explicit negation features \cite{go2009twitter} or Part-of-Speech (POS) tagging \cite{kouloumpis2011twitter}. However, the success of such methods seems to vary significantly with the precise application area and implementation, and such more advanced methods may actually yield worse results than simple unigrams. POS tagging in particular appears to consistently decrease accuracy \cite{agarwal2011sentiment, kouloumpis2011twitter, go2009twitter}.

Frequently, different features are combined (e.g. unigrams + bigrams \cite{go2009twitter}), leading to better overall results.

Sarcasm is frequently used on platforms such as Twitter, but resolving or even detecting sarcasm is very difficult. In \cite{gonzalez2011identifying}, an attempt was made to detect sarcasm in Tweets, but results were not very satisfying. However, humans did not perform much better at this tasks, leading the authors to conclude that extensive knowledge about the context  and participants of a conversation would be required for both humans and algorithms.

Care must also be taken to exclude features which may bias the classification of a message. When searching for tweets containing a certain term, for example, it might be beneficial to exclude this term from the features \cite{go2009twitter}. Otherwise, the overall sentiment for this token might influence the sentiment result of the current tweet, where exactly this sentiment in \textit{relation to the term} is desired.

Semantic analysis can be used to solve many of the aforementioned problems \cite{saif2012semantic}. References to entities are classified into "semantic concepts", allowing for conclusions over broader classes of entities. The authors achieved an improvement of 6.5\% over unigrams and 4.8\% over POS features.

\subsubsection{Machine Learning}

For machine learning, usually some sort of supervised learning is used, where a pre-labelled training set is used for learning (and optionally a second set for evaluation). Once training is complete, new data can be classified. Popular machine learning techniques used in this field include Naive Bayes and Maximum Entropy Classifiers, as well as Support Vector Machines \cite{vinodhini2012sentiment}.

In \cite{go2009twitter}, three algorithms (Naive Bayes, Maximum Entropy and SVM) are compared. Training was done on emoticon data, resulting in accuracy of over 80\% for all three approaches.

Kouloumpis et al. \cite{kouloumpis2011twitter} use AdaBoost as a learning algorithm in their implementation. They found a combination of $n$-grams, pre-tagged words and microblogging-specific features to yield the best results.



Coming up with a suitable dataset to train the machine learning algorithm can be quite challenging on its own. One possibility is the use of manually tagged datasets, but these are usually quite limited in size due to the human effort required to build them. Alternatively, the training set can be automatically tagged using some presumably reliable indicator, such as emoticons \cite{go2009twitter}.


\subsection{Our Approach}

In our work, we used a strongly simplified approach which focuses on individual tokens (unigrams). All aforementioned issues such as negation or sarcasm notwithstanding, the sentiment of a tweet is often determined with reasonable accuracy by a small set of keywords signalling strong emotion.

Semantic or grammatical analysis was deemed to be for too complex, and simplified methods were judged to not provide notable improvements and lead to false positives. We also did not differentiate between objective and subjective valuation, since both are likely relevant for the given use case.

TODO/FIXME: Three different machine learning algorithms were tested: Naive Bayes, IBk and SMO. Training of the support vector machine was significantly slower than for the other algorithms, while results were ??????TODO. 

\subsection{Conclusions}

A frequent observation is that more involved classification methods lead to only marginally better or even worse results (see e.g. \cite{agarwal2011sentiment}).

TODO: extend

\section{Conclusions}


%
% The following two commands are all you need in the
% initial runs of your .tex file to
% produce the bibliography for the citations in your paper.
\bibliographystyle{abbrv}
\bibliography{sigproc}  % sigproc.bib is the name of the Bibliography in this case
% You must have a proper ".bib" file
%  and remember to run:
% latex bibtex latex latex
% to resolve all references
%
% ACM needs 'a single self-contained file'!
%
%APPENDICES are optional
%\balancecolumns
\end{document}
